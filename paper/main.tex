\documentclass{article}
\usepackage{arxiv}

\usepackage[utf8]{inputenc}
\usepackage[english, russian]{babel}
\usepackage[T1]{fontenc}
\usepackage{url}
\usepackage{booktabs}
\usepackage{amsfonts}
\usepackage{nicefrac}
\usepackage{microtype}
\usepackage{lipsum}
\usepackage{graphicx}
\usepackage{natbib}
\usepackage{doi}



\title{Neural SDE: phase trajectories of SDE in the action}

\author{ Papay Ivan\\
	MIPT University \\
	\texttt{papai.id@phystech.edu} \\
	%% examples of more authors
	\And
	Vladimirov Eduard \\
	MIPT University\\
	%% \AND
	%% Coauthor \\
	%% Affiliation \\
	%% Address \\
	%% \texttt{email} \\
	%% \And
	%% Coauthor \\
	%% Affiliation \\
	%% Address \\
	%% \texttt{email} \\
	%% \And
	%% Coauthor \\
	%% Affiliation \\
	%% Address \\
	%% \texttt{email} \\
}
\date{2024 год}

%%\renewcommand{\shorttitle}{\textit{arXiv} Template}

%%% Add PDF metadata to help others organize their library
%%% Once the PDF is generated, you can check the metadata with
%%% $ pdfinfo template.pdf
%\hypersetup{
%pdftitle={A template for the arxiv style},
%pdfsubject={q-bio.NC, q-bio.QM},
%pdfauthor={David S.~Hippocampus, Elias D.~Striatum},
%pdfkeywords={First keyword, Second keyword, More},
%}

\begin{document}
\maketitle

\begin{Аннотация}
    Сама идея использования обыкновенных дифференциальных уравнений("ОДУ" отсюда и далее) далеко не так нова, как могло бы показаться на первый взгляд. Так, примерно с 2017-го года она была использована для создания и теоретического обоснования корректности работы модели Neural ODE. Тем не менее, такой метод был всё ещё слаб в робастном смысле: то есть модель легко подпадала под влияние гауссовского шума, а также была уязвима к состязательным атакам. Модель Neural SDE уже строилась на использовании стохастических дифференциальных уравнений("СДУ" отсюда и далее) и была в этом плане эффективнее своего предшественника. Математический аппарат требовался ещё более серьезный, ведь для вычисления решения СДУ без знаний стохастического анализа, исчисления Ито и Стратоновича обойтись было нельзя. Данная статья предлагает ещё сильнее углубиться в математический аппарат, на котором строится модель Neural SDE. В статье будет рассмотрено, как вычисление фазовых траекторий СДУ обеспечивает качественный прогноз аномалий во временном ряду. Таким образом это предоставит как возможность эффективнее бороться с шумами, так и, в частности, полезный инструмент для упреждения "чёрных лебедей", которые могли бы нарушить корректную работу Neural SDE в виду высокой корреляции элементов анализируемой выборки между собой.
    
\end{Аннотация}


\keywords{SDE \and Stratonovich integral \and More}

\section{Introduction}


\end{document}